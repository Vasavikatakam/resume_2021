%% start of file `template.tex'.
%% Copyright 2006-2013 Xavier Danaux (xdanaux@gmail.com).
%
% This work may be distributed and/or modified under the
% conditions of the LaTeX Project Public License version 1.3c,
% available at http://www.latex-project.org/lppl/.


\documentclass[12pt,a4paper,sans]{moderncv}        % possible options include font size ('10pt', '11pt' and '12pt'), paper size ('a4paper', 'letterpaper', 'a5paper', 'legalpaper', 'executivepaper' and 'landscape') and font family ('sans' and 'roman')

% modern themes
\moderncvstyle{banking}                            % style options are 'casual' (default), 'classic', 'oldstyle' and 'banking'
\moderncvcolor{blue}                                % color options 'blue' (default), 'orange', 'green', 'red', 'purple', 'grey' and 'black'
%\renewcommand{\familydefault}{\sfdefault}         % to set the default font; use '\sfdefault' for the default sans serif font, '\rmdefault' for the default roman one, or any tex font name
%\nopagenumbers{}                                  % uncomment to suppress automatic page numbering for CVs longer than one page

% character encoding
\usepackage[utf8]{inputenc}                       % if you are not using xelatex ou lualatex, replace by the encoding you are using
%\usepackage{CJKutf8}                              % if you need to use CJK to typeset your resume in Chinese, Japanese or Korean

% adjust the page margins
\usepackage[scale=0.75]{geometry}
%\setlength{\hintscolumnwidth}{3cm}                % if you want to change the width of the column with the dates
%\setlength{\makecvtitlenamewidth}{10cm}           % for the 'classic' style, if you want to force the width allocated to your name and avoid line breaks. be careful though, the length is normally calculated to avoid any overlap with your personal info; use this at your own typographical risks...

\usepackage{import}

% personal data
\name{VASAVI}{KATAKAM}                               % optional, remove / comment the line if not wanted
\address{\textbf{Jaggayyapet,Krishna District,
Andhra Pradesh-521175}}{}{}% optional, remove / comment the line if not wanted; the "postcode city" and and "country" arguments can be omitted or provided empty
\phone[mobile]{\textbf{+91 8465998489}}                   % optional, remove / comment the line if not wanted
%\phone[fixed]{01234 123456}                    % optional, remove / comment the line if not wanted
%\phone[fax]{+3~(456)~789~012}                      % optional, remove / comment the line if not wanted
\email{vasavi291198@gmail.com}                               % optional, remove / comment the line if not wanted
%\homepage{www.myname.webs.com}                         % optional, remove / comment the line if not wanted
%\extrainfo{additional information}                 % optional, remove / comment the line if not wanted
%\photo[64pt][0.4pt]{picture}                       % optional, remove / comment the line if not wanted; '64pt' is the height the picture must be resized to, 0.4pt is the thickness of the frame around it (put it to 0pt for no frame) and 'picture' is the name of the picture file
%\quote{Some quote}                                 % optional, remove / comment the line if not wanted

% to show numerical labels in the bibliography (default is to show no labels); only useful if you make citations in your resume
%\makeatletter
%\renewcommand*{\bibliographyitemlabel}{\@biblabel{\arabic{enumiv}}}
%\makeatother
%\renewcommand*{\bibliographyitemlabel}{[\arabic{enumiv}]}% CONSIDER REPLACING THE ABOVE BY THIS

% bibliography with mutiple entries
%\usepackage{multibib}
%\newcites{book,misc}{{Books},{Others}}
%----------------------------------------------------------------------------------
%            content
%----------------------------------------------------------------------------------
\begin{document}
%\begin{CJK*}{UTF8}{gbsn}                          % to typeset your resume in Chinese using CJK
%-----       resume       ---------------------------------------------------------
\makecvtitle
%\section{OBJECTIVE}
\small{Software Developer whose objective is to associate with a company where I can utilize my skills and gain further knowledge while enhancing the company’s productivity and reputation.}

\section{EDUCATION}

%\vspace{3pt}

%\subsection{Academic Qualifications}

%\vspace{5pt}
%\begin{itemize}


{\textbf{Bachelor of Technology}}\hfill{\textbf{2015-2019}}\newline
{Electronics and Communication Engineering}\hfill{CGPA: 8.35/10} \newline
{Indian Institute of Information Technology Sri City} 
\newline

{\textbf{Intermediate(MPC)}}\hfill{\textbf{2013-2015}}\newline
{Sri Chaitanya Junior College,Vijayawada}\hfill{Percentage: 98.1} \newline

{\textbf{High Schooling}}\hfill{\textbf{2012-2013}}\newline
{Zilla Parishad High School,Gandrayi}\hfill{GPA: 9.5/10} %\newline

%\end{itemize}
%\vspace{1pt}

\section{WORK EXPERIENCE}

{\textbf{Technical Associate at DBS Bank}}\hfill{\textbf{July 2019-Present}}\newline
{\textit{Full-time Employee}}\newline

{\textbf{Computer Vision Intern}}\hfill{\textbf{May 2018-July 2018}}\newline
{\textit{Jivass Technologies,Chennai}}\newline
{\small{The focus is on Design and Development of a Computer Vision based module for tracking and automating seedling growth measurements, also developed a user friendly website. }}

{\color{blue}\url{https://docs.google.com/document/d/1Er8sFKQRCou9-rZcvuagjbhLYeh98vP5QxApwS-PftY/edit?usp=sharing}}\newline

{\textbf{Teaching Assistant at IIIT Sri City}}\hfill{\textbf{Jan 2018-Dec 2018}}\newline
{\textit{For Data Structures and Web development courses}}\newline




  % arguments 3 to 6 can be left empty
%\vspace{2pt}

\section{TRAININGS}
\begin{itemize}
\item{\textbf{Deep Learning Specialization}}\hfill{\textbf{Nov 2017-June 2018}}\newline
{\textit{Online Training from Coursera by Prof.AndrewNg}}\newline
\textbf{Certification:}\newline
{\color{blue}\url{https://www.coursera.org/account/accomplishments/certificate/M2LC8HX88VGL}}
\item{\textbf{Machine Learning by Standford University}}\hfill{\textbf{Nov 2017-June 2018}}\newline
{\textit{Online Training from Coursera by Prof.AndrewNg}}
\item{\textbf{Core Java}}\hfill{\textbf{June 2016-July 2016}}\newline
{\textit{Online Training from Internshala}}\newline
\textbf{Certification:}\newline
{\color{blue}\url{https://drive.google.com/file/d/1nYITi9lPyP0GG3rZqwG5xWDsgQgSmwuU/view?usp=sharing}}

\end{itemize}


\vspace{2pt}



\section{NOTABLE PROJECTS}

\vspace{5pt}

\begin{itemize}
\item{\textbf{VIP Customer On boarding}\hfill{\textbf{Dec 2020-Present}} 
\newline\textit{Java,Spring Boot,Hibernate}\newline
\small{Internal Web application targeted for relationship managers to on board VIP customers with the aim of  less manual form filling.}}
\vspace{3pt}
\item{\textbf{Online Account Application Screening}\hfill{\textbf{Mar 2020-Nov 2020}} 
\newline\textit{Java,Spring Boot,BPMN}\newline
\small{An internal portal designed for relationship managers for verifying the case details like companies business details, required documents verification etc., Which is a spring boot application with integrated business process management engine for easily checking the case status at every stage.}}
\vspace{1pt}
\item{\textbf{Online Account Opening System for HK}\hfill{\textbf{Sep 2019-Feb 2020}} 
\newline\textit{Java,Spring Boot}\newline
\small{A restful spring boot based web application aimed to opening accounts for small medium enterprises. Mainly involved in bug fixes in the application. Worked on integrating the rules engine which is needed for deciding the case flow.}}
\vspace{3pt}
\item{\textbf{POC on Generative Adversarial Networks}\hfill{\textbf{Jan 2019-Apr 2019}} 
\newline\textit{Python,Keras}\newline
{\color{blue}\url{https://drive.google.com/file/d/1Ri9uwUpNlg4QJm99s8MLiLWqU96bcc7b/view?usp=sharing}}
\small{Generative adversarial networks (GANs) are deep neural net architectures comprised of two nets,
pitting one against the other (thus the “adversarial”).
These are widely used to generate synthetic
realistic images.I have done POC on this network to generate handwritten Characters.}}
\vspace{3pt}
\item{\textbf{GUI for operating system concepts}\hfill{\textbf{Aug 2016-Dec 2016}} 
\newline\textit{Java,Windowbuilder for GUI}\newline
{\color{blue}\url{https://github.com/Vasavikatakam/OS-SIMULATOR}}
\newline
\small{Implemented a friendly GUI for all the concepts in Operating systems(Scheduling Algorithms to Disk scheduling algorithms) using java.}}

\section{TECHNICAL SKILLS}

\vspace{6pt}

\begin{itemize}

\item \textbf{Programming Languages:} C,C++,Core Java,Python,MySql
\vspace{3pt}
\item \textbf{Operating System:}{Windows,Linux}
\vspace{3pt}
\item\textbf{Frameworks:}{Spring,Hibernate,Django}
\end{itemize}

\section{ACHIEVEMENTS}
\vspace{3pt}

\begin{itemize}
\item{Awarded Super Rookie title as a token of appreciation for my work from DBS.}
\item{Winner in Women hack2hire hackathon which was held by Development bank of Singapore,Hyderabad.}
\item{Received Gold medal for getting first rank in 10th standard.}



\end{itemize}

\section{WEB PRESENCE}

\vspace{6pt}



 \textbf{Github Profile:} 
{\color{blue}\url{https://github.com/Vasavikatakam}}
\vspace{3pt}

\textbf{LinkedIn Profile:}{\color{blue}\url{https://www.linkedin.com/in/vasavi-katakam-092615118/}}






% Publications from a BibTeX file without multibib
%  for numerical labels: \renewcommand{\bibliographyitemlabel}{\@biblabel{\arabic{enumiv}}}% CONSIDER MERGING WITH PREAMBLE PART
%  to redefine the heading string ("Publications"): \renewcommand{\refname}{Articles}
\nocite{*}
\bibliographystyle{plain}
\bibliography{publications}                        % 'publications' is the name of a BibTeX file

% Publications from a BibTeX file using the multibib package
%\section{Publications}
%\nocitebook{book1,book2}
%\bibliographystylebook{plain}
%\bibliographybook{publications}                   % 'publications' is the name of a BibTeX file
%\nocitemisc{misc1,misc2,misc3}
%\bibliographystylemisc{plain}
%\bibliographymisc{publications}                   % 'publications' is the name of a BibTeX file

%-----       letter       ---------------------------------------------------------

\end{document}


%% end of file `template.tex'.
